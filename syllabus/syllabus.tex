\documentclass[10pt]{article}
\usepackage{fullpage}
\usepackage{hyperref}
%\parindent0em\parskip.5em
%\textheight10in
\pagestyle{empty}
\begin{document}
\vspace*{-5em}
\begin{center}
\large \textbf{CMOR 421/521: High Performance Computing}\\[0.5em]
       {Fall 2023 $\cdot$ Rice University}
\end{center}
%\vspace*{-.5em}

\begin{tabular}{rl}
\hline & \\[-.5em]
Lectures:		& T/Th 1pm-2:15pm, Location: DH 1046 \\[.75em]
%
%Web Site: 		& http://www.cmor-faculty.rice.edu/\raisebox{1pt}{$\sim$}caam519 \\[.75em]
%
Instructor:  	& Jesse Chan (jesse.chan@rice.edu), Duncan Hall 2007, (713) 348-6113 \\
			& Office hours: T/Th after class or by appointment.\\[.5em]
%%
%Teaching Assistants: 	& TBD\\%Xiaodi Deng (xd3@rice.edu), Duncan Hall 2106. \\
%			& Office hours: TBD\\ %[.5em]				
\end{tabular}

\paragraph{Overview.} This course will cover high performance computing in scientific computing applications. Topics covered will include serial code optimization, shared and distributed memory programming on multicore and distributed processors, and general purpose graphics processing units (GPGPU). Application interfaces include MPI, OpenMP, and other parallel scientific computing libraries. The course will utilize the C++ and Julia languages. 

\paragraph{References.} Lecture material (which will be made available on Canvas) will be the primary materials for the course. This course has no required textbooks. If you are interested in reference materials that may be useful, the following are good starting points, and should be available online or from the library:
\begin{itemize}
\setlength\itemsep{.1em}
\item An Introduction to Parallel Programming by Peter S. Pacheco. 
\item Bjarne Stroustrup, The C++ Programming Language, 2nd Edition, Addison Wesley, 2013.
\item The Julia manual: \url{https://docs.julialang.org/en/v1/manual/getting-started/}
\item ``The Missing Semester'': \url{https://missing.csail.mit.edu/}
\end{itemize}

\paragraph{Prerequisites.} CMOR 420/520 (Computational Science, also prev.\ known as CAAM 419/519: Computational Science I). Familiarity with scientific computing, Unix/Linux, \LaTeX, C/C++, and Julia.

\paragraph{Technology.} Access to a programmable computer (preferably a laptop) is essential for this course. If you do not have ready access to a computer, please contact the instructor by the end of the first week of classes. 

\paragraph{Grading policy.} Students will be graded based on coding projects, each roughly equivalent to a problem set in terms of complexity and required effort. Students are responsible for ensuring that graders can easily compile, run, and understand their code. You are encouraged to discuss assignments with other students, but the final write-up (including computer code) must be entirely your own work. Likewise, you are encouraged to use online resources (e.g. StackOverflow or LLMs), but directly copying full codes (from online sources or elsewhere) is strictly forbidden. 

\paragraph{Late policy.} Late assignments will be automatically penalized 10\% per day. 

\paragraph{Disability policy.} If you have a documented disability that may affect academic performance, you should make sure this documentation is on file with Disability Support Services (Allen Center, Room 111 / adarice@rice.edu / x5841) to determine the accommodations you need; and meet with me to discuss your accommodation needs.

\paragraph{Rice Honor Code.} It is an honor code violation to turn in code or solutions which have, in all or in part, been copied from another student (including computer codes). It is also an honor code violation to consult solutions to the homework or exams from previous sections of this or similar classes. 

In this course, all students will be held to the standards of the Rice Honor Code (\href{https://cpb-us-e1.wpmucdn.com/blogs.rice.edu/dist/c/490/files/2022/08/Honor-Council-Standard-Definitions-and-Policies.pdf}{link}),
a code that you pledged to honor when you matriculated at this institution. If you are unfamiliar with the details of this code and how it is administered, you should consult the Honor System Handbook at
http://honor.rice.edu/honor-system-handbook/. This handbook outlines the University’s expectations for the integrity of your academic work, the procedures for resolving alleged violations of those expectations,
and the rights and responsibilities of students and faculty members throughout the process.

%\end{center}
\end{document}
